\documentclass[a4paper,twoside]{ociamthesis}
% This one will format for one-sided binding (ie left margin > right margin; no extra blank pages):
%\documentclass[a4paper]{ociamthesis}
% This one will format for PDF output (ie equal margins, no extra blank pages):
%\documentclass[a4paper,nobind]{ociamthesis} 

%%%%% SELECT YOUR DRAFT OPTIONS
% Three options going on here; use in any combination.  But remember to turn the first two off before
% generating a PDF to send to the printer!

% This adds a "DRAFT" footer to every normal page.  (The first page of each chapter is not a "normal" page.)
\fancyfoot[C]{\emph{DRAFT version \today}}  

% This highlights (in blue) corrections marked with (for words) \mccorrect{blah} or (for whole
% paragraphs) \begin{mccorrection} . . . \end{mccorrection}.  This can be useful for sending a PDF of
% your corrected thesis to your examiners for review.  Turn it off, and the blue disappears.
\correctionstrue{}

% Standard packages and definitions
\usepackage{tabularx}
\usepackage{subcaption}
\usepackage{footnote}
\usepackage{lipsum}  
\usepackage[version=4]{mhchem}
\usepackage{float}
\usepackage{bold-extra}
\usepackage{abstract}

\captionsetup{format=hang}
\newcolumntype{C}{>{\centering\arraybackslash}X}
\newcolumntype{L}{>{\raggedright\arraybackslash}X}
\newcolumntype{R}{>{\raggedleft\arraybackslash}X}

\input{colors.tex}

%%%%% BIBLIOGRAPHY SETUP
\usepackage[
	style=ieee,
	sortcites=false,
	backend=biber,
	doi=true,
	isbn=true,
	maxcitenames=2,
	mincitenames=1,
	uniquelist=true,
	abbreviate,
	maxbibnames=99,
	citestyle=numeric-comp, % Multiple citations delimited by ", "
	% autocite=superscript
]{biblatex}

% Bold authors in bibliography
\DeclareNameWrapperFormat{sortname}{\mkbibbold{#1}}
\DeclareNameWrapperAlias{author}{sortname}
\DeclareNameWrapperAlias{editor}{sortname}
\DeclareNameWrapperAlias{translator}{sortname}
% \renewcommand\citepunct{, } 

% Pretty references
\usepackage[
    pdfpagelabels,
	    colorlinks=true,
    citecolor=blue,
    urlcolor=blue,
    linkcolor=black
]{hyperref} 

\usepackage{bookmark}
	% for linking between references, figures, TOC, etc in the pdf document

% Equation spacing
\expandafter\def\expandafter\normalsize\expandafter{%
    \normalsize%
    \setlength\abovedisplayskip{12pt plus 4pt minus 6pt}%
    \setlength\belowdisplayskip{12pt plus 4pt minus 8pt}%
    % \setlength\abovedisplayshortskip{0pt}%
    % \setlength\belowdisplayshortskip{0pt}%
}

% Change this to the name of your .bib file (usually exported from a citation manager like Zotero or EndNote).
\addbibresource{ref.bib}

\newcommand*{\bibtitle}{References}

% Pages shouldn't have numbering here
\pagenumbering{gobble} 

% Uncomment this if you want equation numbers per section (2.3.12), instead of per chapter (2.18):
%\numberwithin{equation}{subsection}

%%%%% THESIS / TITLE PAGE INFORMATION
% Everybody needs to complete the following:
\title{Suitable Impressive Thesis Title}
\author{John Doe}
\college{Institute of Philology}
\degree{Dr.~rer. nat.}
\degreedate{}


%%%%% YOUR OWN PERSONAL MACROS
% This is a good place to dump your own LaTeX macros as they come up.
\newcommand{\er}[1]{Eq.~\eqref{eq:#1}}
\newcommand{\erp}[2]{Eqns.~\eqref{eq:#1}\eqref{eq:#2}}
\newcommand{\be}[1]{\textsubscript{#1}}  % easier textsubscripts
\newcommand{\ab}[1]{\textsubscript{#1}}  % easier textsubscripts
\renewcommand{\th}{\textsuperscript{th}} % ex: I won 4\th place
\newcommand{\nd}{\textsuperscript{nd}}
\renewcommand{\st}{\textsuperscript{st}}
\newcommand{\rd}{\textsuperscript{rd}}
\newcommand{\vecc}[1]{\mathbf{#1}}
\newcommand{\partialdifof}[2]{\frac{\partial #1}{\partial #2}}
\newcommand{\bt}[1]{\texttt{\textbf{#1}}}

% Custom defs
\newcommand{\isp}{$I_{\mathrm{sp}}$}
\newcommand{\ispm}{I_{\mathrm{sp}}}

\setlength{\jot}{10pt}
\usepackage{xltabular}
\newcolumntype{R}{>{\raggedleft\arraybackslash}X}



\usepackage{fontspec}
\usepackage{unicode-math}
\setmainfont[Scale=0.93]{TeX Gyre Schola}
\setmathfont[Scale=0.93]{TeX Gyre Schola Math}

%%%%% THE ACTUAL DOCUMENT STARTS HERE
\begin{document}

%%%%% CHOOSE YOUR LINE SPACING HERE
% This is the official option.  Use it for your submission copy and library copy:
% \setlength{\textbaselineskip}{16pt plus2pt}
\setlength{\textbaselineskip}{18pt plus1pt minus1pt}
% This is closer spacing (about 1.5-spaced) that you might prefer for your personal copies:
%\setlength{\textbaselineskip}{18pt plus2pt minus1pt}

% You can set the spacing here for the roman-numbered pages (acknowledgements, table of contents, etc.)
\setlength{\frontmatterbaselineskip}{17pt plus1pt minus1pt}

% Leave this line alone; it gets things started for the real document.
\setlength{\baselineskip}{\textbaselineskip}

%%%%% CHOOSE YOUR SECTION NUMBERING DEPTH HERE
% You have two choices.  First, how far down are sections numbered?  (Below that, they're named but
% don't get numbers.)  Second, what level of section appears in the table of contents?  These don't have
% to match: you can have numbered sections that don't show up in the ToC, or unnumbered sections that
% do.  Throughout, 0 = chapter; 1 = section; 2 = subsection; 3 = subsubsection, 4 = paragraph...

% The level that gets a number:
\setcounter{secnumdepth}{2}
% The level that shows up in the ToC:
\setcounter{tocdepth}{2}

% JEM: Pages are roman numbered from here, though page numbers are invisible until ToC.  This is in
% keeping with most typesetting conventions.

% Title page is created here
\maketitle

%%%%% DEDICATION -- If you'd like one, un-comment the following.
%\begin{dedication}
%This thesis is dedicated to\\
%someone\\
%for some special reason\\
%\end{dedication}

%%%%% ACKNOWLEDGEMENTS -- Nothing to do here except comment out if you don't want it.
\begin{acknowledgements}
 	\lipsum[1]
\end{acknowledgements}

\newpage \ \newpage

%%%%% ABSTRACT -- Nothing to do here except comment out if you don't want it.
\renewcommand{\abstractname}{Kurzfassung}
\begin{abstract}
	Dies ist ein deutscher Abstrakt.
\end{abstract}
\newpage \ \newpage
\renewcommand{\abstractname}{Abstract}
\begin{abstract}
    \lipsum[2]
\end{abstract}
\newpage \ \newpage

\dominitoc % include a mini table of contents

% This aligns the bottom of the text of each page.  It generally makes things look better.
\flushbottom

\pagenumbering{roman}

% This is where the whole-document ToC appears:
\tableofcontents

% List of Symbols
\chapter*{List of Symbols}
\mtcaddchapter[List of Symbols]

\begin{center}
\begin{tabular}{cll}

	\textbf{Symbol} & \textbf{Description} & \textbf{Unit}  \\
	\(\vecc{a}\) & acceleration & \si{\metre\per\second\squared}  \\
	\(\vecc{F}\) & force & \si{\newton}  \\
	\(\dot{m}\) & mass-flow & \si{\kilogram\per\second}  \\
	\(\vecc{x}\) & position & \si{m}  \\

\end{tabular}
\end{center}


% List of Abbreviations
\chapter*{List of Abbreviations}
\mtcaddchapter[List of Abbreviations]

\begin{center}
\begin{tabular}{ll}

	\textbf{Abbr.} & \textbf{Description} \\
	AIREBO & Adaptive intermolecular reactive bond order \\
	CEX    & Charge exchange \\
	CP     & Chemical propulsion \\

\end{tabular}
\end{center}


\listoffigures
\mtcaddchapter
% \mtcaddchapter is needed when adding a non-chapter (but chapter-like) entity to avoid confusing minitoc

% Uncomment to generate a list of tables:
%\listoftables
%	\mtcaddchapter

%%%%% CHAPTERS
% Add or remove any chapters you'd like here, by file name (excluding '.tex'):
\flushbottom

\begin{savequote}[8cm]
  Starting from Day 0 became my friend.
  
  \qauthor{--- D. Goggins}
\end{savequote}

\chapter{\label{ch:introduction}Introduction}
\pagenumbering{arabic}

\minitoc

\lipsum[3]

\begin{equation}\label{eq:thrust}
T = \dot{m} u_e
\end{equation}

\begin{figure}[t!]
  \centering
  \includegraphics[width=\textwidth]{fig/1.introduction/final/ep_schematic.pdf}
  \caption{Schematic of an electrostatic propulsion device.}
  \label{fig:ep_schematic}
\end{figure}

A schematic of the working principle of an electrostatic thruster is
illustrated in Figure~\ref{fig:ep_schematic}. 

\section{Background and Motivation}\label{sec:background-and-motivation}

The lifetime of most EP devices is limited by erosion becoming so
severe, that the accelerating mechanism does not work properly anymore.

\section{Literature Review}\label{sec:literature-review}

Because there is limited literature available on MD simulations of EP plasma
sputtering according to~\textcite{jackson2019}, this literature review includes
publications from various fields where similar analyses were
conducted~\autocite{harvey2007}.

\begin{savequote}[8cm]
  Keep your eyes on the stars, but remember to keep your feet on the ground.
  
  \qauthor{--- T. Roosevelt}
\end{savequote}

\chapter{\label{ch:theory-of-sputtering}Theory of ion beam sputtering}

\minitoc


Surfaces subjected to ion irradiation undergo a cascade of energetic
displacement on the atomic scale.

\section{Test section}\label{sec:test-section}

%% APPENDICES %% 
% Starts lettered appendices, adds a heading in table of contents, and adds a
%    page that just says "Appendices" to signal the end of your main text.
% \startappendices
% % Add or remove any appendices you'd like here:
% \include{text/appendix-1}


% %%%%% REFERENCES
\printbibliography


\end{document}
